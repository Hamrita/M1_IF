% Options for packages loaded elsewhere
\PassOptionsToPackage{unicode}{hyperref}
\PassOptionsToPackage{hyphens}{url}
%
\documentclass[
]{article}
\usepackage{lmodern}
\usepackage{amssymb,amsmath}
\usepackage{ifxetex,ifluatex}
\ifnum 0\ifxetex 1\fi\ifluatex 1\fi=0 % if pdftex
  \usepackage[T1]{fontenc}
  \usepackage[utf8]{inputenc}
  \usepackage{textcomp} % provide euro and other symbols
\else % if luatex or xetex
  \usepackage{unicode-math}
  \defaultfontfeatures{Scale=MatchLowercase}
  \defaultfontfeatures[\rmfamily]{Ligatures=TeX,Scale=1}
\fi
% Use upquote if available, for straight quotes in verbatim environments
\IfFileExists{upquote.sty}{\usepackage{upquote}}{}
\IfFileExists{microtype.sty}{% use microtype if available
  \usepackage[]{microtype}
  \UseMicrotypeSet[protrusion]{basicmath} % disable protrusion for tt fonts
}{}
\makeatletter
\@ifundefined{KOMAClassName}{% if non-KOMA class
  \IfFileExists{parskip.sty}{%
    \usepackage{parskip}
  }{% else
    \setlength{\parindent}{0pt}
    \setlength{\parskip}{6pt plus 2pt minus 1pt}}
}{% if KOMA class
  \KOMAoptions{parskip=half}}
\makeatother
\usepackage{xcolor}
\IfFileExists{xurl.sty}{\usepackage{xurl}}{} % add URL line breaks if available
\IfFileExists{bookmark.sty}{\usepackage{bookmark}}{\usepackage{hyperref}}
\hypersetup{
  pdftitle={Les vecteurs},
  hidelinks,
  pdfcreator={LaTeX via pandoc}}
\urlstyle{same} % disable monospaced font for URLs
\usepackage[margin=1in]{geometry}
\usepackage{color}
\usepackage{fancyvrb}
\newcommand{\VerbBar}{|}
\newcommand{\VERB}{\Verb[commandchars=\\\{\}]}
\DefineVerbatimEnvironment{Highlighting}{Verbatim}{commandchars=\\\{\}}
% Add ',fontsize=\small' for more characters per line
\usepackage{framed}
\definecolor{shadecolor}{RGB}{248,248,248}
\newenvironment{Shaded}{\begin{snugshade}}{\end{snugshade}}
\newcommand{\AlertTok}[1]{\textcolor[rgb]{0.94,0.16,0.16}{#1}}
\newcommand{\AnnotationTok}[1]{\textcolor[rgb]{0.56,0.35,0.01}{\textbf{\textit{#1}}}}
\newcommand{\AttributeTok}[1]{\textcolor[rgb]{0.77,0.63,0.00}{#1}}
\newcommand{\BaseNTok}[1]{\textcolor[rgb]{0.00,0.00,0.81}{#1}}
\newcommand{\BuiltInTok}[1]{#1}
\newcommand{\CharTok}[1]{\textcolor[rgb]{0.31,0.60,0.02}{#1}}
\newcommand{\CommentTok}[1]{\textcolor[rgb]{0.56,0.35,0.01}{\textit{#1}}}
\newcommand{\CommentVarTok}[1]{\textcolor[rgb]{0.56,0.35,0.01}{\textbf{\textit{#1}}}}
\newcommand{\ConstantTok}[1]{\textcolor[rgb]{0.00,0.00,0.00}{#1}}
\newcommand{\ControlFlowTok}[1]{\textcolor[rgb]{0.13,0.29,0.53}{\textbf{#1}}}
\newcommand{\DataTypeTok}[1]{\textcolor[rgb]{0.13,0.29,0.53}{#1}}
\newcommand{\DecValTok}[1]{\textcolor[rgb]{0.00,0.00,0.81}{#1}}
\newcommand{\DocumentationTok}[1]{\textcolor[rgb]{0.56,0.35,0.01}{\textbf{\textit{#1}}}}
\newcommand{\ErrorTok}[1]{\textcolor[rgb]{0.64,0.00,0.00}{\textbf{#1}}}
\newcommand{\ExtensionTok}[1]{#1}
\newcommand{\FloatTok}[1]{\textcolor[rgb]{0.00,0.00,0.81}{#1}}
\newcommand{\FunctionTok}[1]{\textcolor[rgb]{0.00,0.00,0.00}{#1}}
\newcommand{\ImportTok}[1]{#1}
\newcommand{\InformationTok}[1]{\textcolor[rgb]{0.56,0.35,0.01}{\textbf{\textit{#1}}}}
\newcommand{\KeywordTok}[1]{\textcolor[rgb]{0.13,0.29,0.53}{\textbf{#1}}}
\newcommand{\NormalTok}[1]{#1}
\newcommand{\OperatorTok}[1]{\textcolor[rgb]{0.81,0.36,0.00}{\textbf{#1}}}
\newcommand{\OtherTok}[1]{\textcolor[rgb]{0.56,0.35,0.01}{#1}}
\newcommand{\PreprocessorTok}[1]{\textcolor[rgb]{0.56,0.35,0.01}{\textit{#1}}}
\newcommand{\RegionMarkerTok}[1]{#1}
\newcommand{\SpecialCharTok}[1]{\textcolor[rgb]{0.00,0.00,0.00}{#1}}
\newcommand{\SpecialStringTok}[1]{\textcolor[rgb]{0.31,0.60,0.02}{#1}}
\newcommand{\StringTok}[1]{\textcolor[rgb]{0.31,0.60,0.02}{#1}}
\newcommand{\VariableTok}[1]{\textcolor[rgb]{0.00,0.00,0.00}{#1}}
\newcommand{\VerbatimStringTok}[1]{\textcolor[rgb]{0.31,0.60,0.02}{#1}}
\newcommand{\WarningTok}[1]{\textcolor[rgb]{0.56,0.35,0.01}{\textbf{\textit{#1}}}}
\usepackage{graphicx,grffile}
\makeatletter
\def\maxwidth{\ifdim\Gin@nat@width>\linewidth\linewidth\else\Gin@nat@width\fi}
\def\maxheight{\ifdim\Gin@nat@height>\textheight\textheight\else\Gin@nat@height\fi}
\makeatother
% Scale images if necessary, so that they will not overflow the page
% margins by default, and it is still possible to overwrite the defaults
% using explicit options in \includegraphics[width, height, ...]{}
\setkeys{Gin}{width=\maxwidth,height=\maxheight,keepaspectratio}
% Set default figure placement to htbp
\makeatletter
\def\fps@figure{htbp}
\makeatother
\setlength{\emergencystretch}{3em} % prevent overfull lines
\providecommand{\tightlist}{%
  \setlength{\itemsep}{0pt}\setlength{\parskip}{0pt}}
\setcounter{secnumdepth}{-\maxdimen} % remove section numbering

\title{Les vecteurs}
\author{}
\date{\vspace{-2.5em}}

\begin{document}
\maketitle

\hypertarget{cruxe9ation-dun-vecteur}{%
\subsection{Création d'un vecteur}\label{cruxe9ation-dun-vecteur}}

Pour créer un vecteur, la façon la plus simple est utiliser la fonction
\textbf{\texttt{c()}}.

\begin{Shaded}
\begin{Highlighting}[]
\CommentTok{# Exemple x=[1 4 5]}
\NormalTok{x=}\KeywordTok{c}\NormalTok{(}\DecValTok{1}\NormalTok{,}\DecValTok{4}\NormalTok{,}\DecValTok{5}\NormalTok{)}
\NormalTok{x}
\end{Highlighting}
\end{Shaded}

\begin{verbatim}
[1] 1 4 5
\end{verbatim}

Pour créer une série régulière, \texttt{R} propose les fonctions
\texttt{:}, \texttt{seq()} et la fonction \texttt{rep()}.

\begin{Shaded}
\begin{Highlighting}[]
\CommentTok{# exemple série de 1 à 10}
\DecValTok{1}\OperatorTok{:}\DecValTok{10}
\end{Highlighting}
\end{Shaded}

\begin{verbatim}
 [1]  1  2  3  4  5  6  7  8  9 10
\end{verbatim}

\begin{Shaded}
\begin{Highlighting}[]
\CommentTok{# série de 20 à 2}
\DecValTok{20}\OperatorTok{:}\DecValTok{2}
\end{Highlighting}
\end{Shaded}

\begin{verbatim}
 [1] 20 19 18 17 16 15 14 13 12 11 10  9  8  7  6  5  4  3  2
\end{verbatim}

\begin{Shaded}
\begin{Highlighting}[]
\FloatTok{1.2}\OperatorTok{:}\FloatTok{5.7}  
\end{Highlighting}
\end{Shaded}

\begin{verbatim}
[1] 1.2 2.2 3.2 4.2 5.2
\end{verbatim}

\begin{Shaded}
\begin{Highlighting}[]
\CommentTok{# générer une séquence à l'aide de la fonction `seq()`}
\CommentTok{# taper `?seq` pour consulter l'aide de la fonction}
\KeywordTok{seq}\NormalTok{(}\DataTypeTok{from=}\DecValTok{1}\NormalTok{,}\DataTypeTok{to=}\DecValTok{10}\NormalTok{)}
\end{Highlighting}
\end{Shaded}

\begin{verbatim}
 [1]  1  2  3  4  5  6  7  8  9 10
\end{verbatim}

\begin{Shaded}
\begin{Highlighting}[]
\CommentTok{# introduisons les arguments `by' et `len`}
\KeywordTok{seq}\NormalTok{(}\DecValTok{1}\NormalTok{,}\DecValTok{3}\NormalTok{,}\DataTypeTok{by=}\FloatTok{0.5}\NormalTok{)  }\CommentTok{# création d'une séquence de 1 à 3 avec un pas de 0.5}
\end{Highlighting}
\end{Shaded}

\begin{verbatim}
[1] 1.0 1.5 2.0 2.5 3.0
\end{verbatim}

\begin{Shaded}
\begin{Highlighting}[]
\KeywordTok{seq}\NormalTok{(}\DecValTok{1}\NormalTok{,}\DecValTok{3}\NormalTok{, }\DataTypeTok{len=}\DecValTok{5}\NormalTok{) }\CommentTok{# création d'une séquence de 1 à 2 de longueur 10}
\end{Highlighting}
\end{Shaded}

\begin{verbatim}
[1] 1.0 1.5 2.0 2.5 3.0
\end{verbatim}

\begin{Shaded}
\begin{Highlighting}[]
\CommentTok{# vecteur contenant des répétitions}
\KeywordTok{rep}\NormalTok{(}\DecValTok{5}\NormalTok{,}\DataTypeTok{times=}\DecValTok{5}\NormalTok{)}
\end{Highlighting}
\end{Shaded}

\begin{verbatim}
[1] 5 5 5 5 5
\end{verbatim}

\begin{Shaded}
\begin{Highlighting}[]
\KeywordTok{rep}\NormalTok{(}\KeywordTok{c}\NormalTok{(}\DecValTok{1}\NormalTok{,}\DecValTok{2}\NormalTok{),}\DecValTok{6}\NormalTok{) }\CommentTok{# répéter (1,2) 6 fois.}
\end{Highlighting}
\end{Shaded}

\begin{verbatim}
 [1] 1 2 1 2 1 2 1 2 1 2 1 2
\end{verbatim}

\begin{Shaded}
\begin{Highlighting}[]
\KeywordTok{rep}\NormalTok{(}\KeywordTok{c}\NormalTok{(}\DecValTok{1}\NormalTok{,}\DecValTok{2}\NormalTok{),}\DataTypeTok{each=}\DecValTok{3}\NormalTok{) }\CommentTok{# répéter chaque éléments du vecteur (1,2) 3 fois}
\end{Highlighting}
\end{Shaded}

\begin{verbatim}
[1] 1 1 1 2 2 2
\end{verbatim}

\begin{Shaded}
\begin{Highlighting}[]
\KeywordTok{rep}\NormalTok{(}\KeywordTok{c}\NormalTok{(}\DecValTok{1}\NormalTok{,}\DecValTok{2}\NormalTok{),}\DataTypeTok{each=}\DecValTok{3}\NormalTok{,}\DataTypeTok{len=}\DecValTok{10}\NormalTok{)}
\end{Highlighting}
\end{Shaded}

\begin{verbatim}
 [1] 1 1 1 2 2 2 1 1 1 2
\end{verbatim}

\hypertarget{indexation-et-accuxe9s}{%
\subsection{Indexation et accés}\label{indexation-et-accuxe9s}}

\begin{Shaded}
\begin{Highlighting}[]
\CommentTok{# indexation}
\NormalTok{x=}\KeywordTok{seq}\NormalTok{(}\DecValTok{1}\NormalTok{,}\DecValTok{5}\NormalTok{,}\DataTypeTok{len=}\DecValTok{15}\NormalTok{)}
\NormalTok{x}
\end{Highlighting}
\end{Shaded}

\begin{verbatim}
 [1] 1.000000 1.285714 1.571429 1.857143 2.142857 2.428571 2.714286
 [8] 3.000000 3.285714 3.571429 3.857143 4.142857 4.428571 4.714286
[15] 5.000000
\end{verbatim}

\begin{Shaded}
\begin{Highlighting}[]
\NormalTok{x[}\DecValTok{5}\NormalTok{] }\CommentTok{# extraction de la 5ième valeur}
\end{Highlighting}
\end{Shaded}

\begin{verbatim}
[1] 2.142857
\end{verbatim}

\begin{Shaded}
\begin{Highlighting}[]
\NormalTok{x[}\OperatorTok{-}\DecValTok{4}\NormalTok{] }\CommentTok{# éliminer la valeur du 4ième index}
\end{Highlighting}
\end{Shaded}

\begin{verbatim}
 [1] 1.000000 1.285714 1.571429 2.142857 2.428571 2.714286 3.000000
 [8] 3.285714 3.571429 3.857143 4.142857 4.428571 4.714286 5.000000
\end{verbatim}

\begin{Shaded}
\begin{Highlighting}[]
\NormalTok{x[x}\OperatorTok{>=}\FloatTok{1.2}\NormalTok{] }\CommentTok{# extraire les valeurs >= à 1.2}
\end{Highlighting}
\end{Shaded}

\begin{verbatim}
 [1] 1.285714 1.571429 1.857143 2.142857 2.428571 2.714286 3.000000
 [8] 3.285714 3.571429 3.857143 4.142857 4.428571 4.714286 5.000000
\end{verbatim}

\begin{Shaded}
\begin{Highlighting}[]
\NormalTok{x[}\DecValTok{1}\OperatorTok{:}\DecValTok{4}\NormalTok{]      }\CommentTok{# extraire les éléments de x de la première position à la 4ième.}
\end{Highlighting}
\end{Shaded}

\begin{verbatim}
[1] 1.000000 1.285714 1.571429 1.857143
\end{verbatim}

\begin{Shaded}
\begin{Highlighting}[]
\NormalTok{x[}\OperatorTok{-}\KeywordTok{c}\NormalTok{(}\DecValTok{1}\NormalTok{,}\DecValTok{5}\NormalTok{)]  }\CommentTok{# afficher toutes les valeurs de x suaf la 1ière et la 5ième valeur.}
\end{Highlighting}
\end{Shaded}

\begin{verbatim}
 [1] 1.285714 1.571429 1.857143 2.428571 2.714286 3.000000 3.285714
 [8] 3.571429 3.857143 4.142857 4.428571 4.714286 5.000000
\end{verbatim}

\hypertarget{statistiques-uxe9luxe9mentaires}{%
\subsection{Statistiques
élémentaires}\label{statistiques-uxe9luxe9mentaires}}

\begin{Shaded}
\begin{Highlighting}[]
\CommentTok{# la longueur d'un vecteur est obtenu à l'aide de la fonction length}
\KeywordTok{length}\NormalTok{(x)}
\end{Highlighting}
\end{Shaded}

\begin{verbatim}
[1] 15
\end{verbatim}

\begin{Shaded}
\begin{Highlighting}[]
\CommentTok{# maximum}
\KeywordTok{max}\NormalTok{(x)}
\end{Highlighting}
\end{Shaded}

\begin{verbatim}
[1] 5
\end{verbatim}

\begin{Shaded}
\begin{Highlighting}[]
\KeywordTok{min}\NormalTok{(x)}
\end{Highlighting}
\end{Shaded}

\begin{verbatim}
[1] 1
\end{verbatim}

\begin{Shaded}
\begin{Highlighting}[]
\KeywordTok{sum}\NormalTok{(x) }\CommentTok{# la somme des val de x}
\end{Highlighting}
\end{Shaded}

\begin{verbatim}
[1] 45
\end{verbatim}

\begin{Shaded}
\begin{Highlighting}[]
\KeywordTok{cumsum}\NormalTok{(x) }\CommentTok{# la somme cumulée}
\end{Highlighting}
\end{Shaded}

\begin{verbatim}
 [1]  1.000000  2.285714  3.857143  5.714286  7.857143 10.285714 13.000000
 [8] 16.000000 19.285714 22.857143 26.714286 30.857143 35.285714 40.000000
[15] 45.000000
\end{verbatim}

\begin{Shaded}
\begin{Highlighting}[]
\KeywordTok{prod}\NormalTok{(x) }\CommentTok{# produit des val de x}
\end{Highlighting}
\end{Shaded}

\begin{verbatim}
[1] 3112378
\end{verbatim}

\begin{Shaded}
\begin{Highlighting}[]
\KeywordTok{cumprod}\NormalTok{(x) }\CommentTok{# produit cumulé}
\end{Highlighting}
\end{Shaded}

\begin{verbatim}
 [1] 1.000000e+00 1.285714e+00 2.020408e+00 3.752187e+00 8.040400e+00
 [6] 1.952669e+01 5.300100e+01 1.590030e+02 5.224385e+02 1.865852e+03
[11] 7.196856e+03 2.981555e+04 1.320403e+05 6.224756e+05 3.112378e+06
\end{verbatim}

\begin{Shaded}
\begin{Highlighting}[]
\KeywordTok{which.min}\NormalTok{(x)     }\CommentTok{# donne la position pour laquelle x est minimale}
\end{Highlighting}
\end{Shaded}

\begin{verbatim}
[1] 1
\end{verbatim}

\begin{Shaded}
\begin{Highlighting}[]
\KeywordTok{which}\NormalTok{(x}\OperatorTok{>}\FloatTok{3.3}\NormalTok{)  }\CommentTok{# donne les positions pour lesquelles les valeurs de x sont supèrieur à 3.3}
\end{Highlighting}
\end{Shaded}

\begin{verbatim}
[1] 10 11 12 13 14 15
\end{verbatim}

\begin{Shaded}
\begin{Highlighting}[]
\CommentTok{# la moyenne empirique}
\KeywordTok{mean}\NormalTok{(x)}
\end{Highlighting}
\end{Shaded}

\begin{verbatim}
[1] 3
\end{verbatim}

\begin{Shaded}
\begin{Highlighting}[]
\CommentTok{# multiplication d'un vecteur par un scalaire}
\DecValTok{3}\OperatorTok{*}\NormalTok{x}
\end{Highlighting}
\end{Shaded}

\begin{verbatim}
 [1]  3.000000  3.857143  4.714286  5.571429  6.428571  7.285714  8.142857
 [8]  9.000000  9.857143 10.714286 11.571429 12.428571 13.285714 14.142857
[15] 15.000000
\end{verbatim}

\begin{Shaded}
\begin{Highlighting}[]
\CommentTok{# x'x produit scalaire}
\KeywordTok{crossprod}\NormalTok{(x)}
\end{Highlighting}
\end{Shaded}

\begin{verbatim}
         [,1]
[1,] 157.8571
\end{verbatim}

\begin{Shaded}
\begin{Highlighting}[]
\CommentTok{# ou encore}
\KeywordTok{t}\NormalTok{(x)}\OperatorTok\NormalTok{x}
\end{Highlighting}
\end{Shaded}

\begin{verbatim}
         [,1]
[1,] 157.8571
\end{verbatim}

\end{document}
